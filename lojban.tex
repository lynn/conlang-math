\documentclass[12pt,a4paper]{article}
\usepackage{amsmath,amsthm,parskip,xcolor}
\usepackage{csquotes}
\usepackage{lmodern}

% Korekti teoreman interspacon.
{\makeatletter\@for\theoremstyle:=definition,remark,plain\do{\expandafter\g@addto@macro\csname th@\theoremstyle\endcsname{\addtolength\thm@preskip\parskip}}}

% Ruĝa teksto.
\newcommand{\red}[1]{{\color{red}#1}}

% Elŝalti paĝnombrojn.
\pagenumbering{gobble}

\usepackage[compact,explicit]{titlesec}
\titleformat{\section}[runin]{\Large\bfseries}{}{0pt}{ni'oni'oni'o \thesection{}mo'o #1}
\titleformat{\subsection}[runin]{\large\bfseries}{}{0pt}{ni'oni'o \arabic{subsection}mai #1}

% Teorema kaj difina ĉirkaŭaĵo.
\theoremstyle{plain}\newtheorem*{mythm}{ni'o ju'a}
\theoremstyle{definition}\newtheorem*{mydef}{ni'o ca'e}

\begin{document}
\section{tefsujme'o}
\subsection{lo tefsujme'o ku mo?}
\begin{mydef}
	\textbf{tefsujme'o (cei broda)} fa ro mekso poi simsumji su'ono za'e ``\textbf{sajysle}'' noi ro ke'a ca'e pilji lo nonfrica lo su'ono sinxa, noi ca'e \textbf{snicne}
\end{mydef}

ni'o mu'a la'e lu ``$3x^2 + (-2x) + 7$'' li'u broda .i ri se pagbu ci sajysle je pa snicne no'u li $x$ (to sa'unai lo pamoi sajysle ku pilji li $3$ li $x$ li $x$ ke'u toi)

ni'o ca'e lo ``\textbf{fancu'alge}'' be lo broda goi py bei lo namcu goi xy cu te pruce xy lo nu xy pa'a basti lo sinxa pe py lo nu kanji

ni'o mu'a fancu'alge li $3x^2 + (-2x) + 7$ li $x=4$ fa li \[
3 \cdot \red{4}^2 + (-2 \cdot \red{4}) + 7 = 48 - 8 + 7 = 47.
\]
ni'o baze'e casnu lo broda pe lo pa po'o snicne no'u li $x$

\subsection{la kurtenfa mekso zo'u}
\begin{mydef}
	za'e \textbf{nonco'e} lo broda goi py fa ro namcu poi te fancu'alge py li 0 .i mu'a li 3 nonco'e li $x^2-9$ ni'i lodu'u li $3^2 - 9$ mintu li $0$
\end{mydef}

\begin{mydef}
	za'e \textbf{kurtefco'e} fa ro broda poi su'o nonfrica goi li $a$ su'o namcu goi li $b$ su'o namcu goi li $c$ zo'u: ke'a mintu li $ax^2 + bx + c$.
\end{mydef}

\begin{mythm}
	lo nonco'e goi li $x$ be li $a x^2 + b x + c$ cu jai do'e mekso fai li \[
	\boxed{x = \frac{-b \pm \sqrt{b^2-4ac}}{2a}.}
	\]
\end{mythm}

\begin{proof}
	In the equation $a x^2 + b x + c = 0$, divide both sides by $a$: \[
	x^2 + \frac{b}{a} x + \frac{c}{a} = 0.
	\]
	Now add $\left( \frac{b}{2a} \right)^2 - \frac{c}{a}$ to both sides: \[
	x^2 + \frac{b}{a} x + \left( \frac{b}{2a} \right)^2 = \left( \frac{b}{2a} \right)^2 - \frac{c}{a}.
	\]
	Recognize the left side as $(x + \frac{b}{2a})^2$ and simplify the right side. \[
	\left(x + \frac{b}{2a}\right)^2 = \frac{b^2 - 4ac}{4a^2}
	\]
	We know that $p^2 = q$ precisely when $p = \pm \sqrt{q}$. So we conclude: \[
	x + \frac{b}{2a} = \sqrt{\frac{b^2 - 4ac}{4a^2}},
	\]
	which we can write as \[
	\boxed{x = \frac{-b \pm \sqrt{b^2 - 4ac}}{2a}.} \qedhere
	\]
\end{proof}
\end{document}
