\documentclass[12pt,a4paper]{article}
\usepackage{amsmath,amsthm,parskip,xcolor}
\usepackage[esperanto]{babel}

% Teorema kaj difina ĉirkaŭaĵo.
\theoremstyle{plain}\newtheorem*{mythm}{Teoremo}
\theoremstyle{definition}\newtheorem*{mydef}{Difino}

% Korekti teoreman interspacon.
{\makeatletter\@for\theoremstyle:=definition,remark,plain\do{\expandafter\g@addto@macro\csname th@\theoremstyle\endcsname{\addtolength\thm@preskip\parskip}}}

% Ruĝa teksto.
\newcommand{\red}[1]{{\color{red}#1}}

% Elŝalti paĝnombrojn.
\pagenumbering{gobble}

\begin{document}
\section{Polinomoj}
\subsection{Kio estas polinomoj?}
\begin{mydef}
	\textbf{Polinomo} estas matematika esprimo, kiu estas finia sumo de \textbf{termoj}. ^Ciu termo estas produto de ne-nula nombro, kaj (opcie) iuj simboloj a^u literoj. La simboloj nomi^gas \textbf{variabloj}.
\end{mydef}

Ekzemple, ``$3x^2 + (-2x) + 7$'' estas polinomo. ^Gi konsistas el tri termoj, kaj unu variablo, $x$. (La unua termo estas la produto de $3$, $x$, kaj denove $x$.)

Ni povas ``\textbf{elkalkuli}'' polinomon: elekti nombran valoron por ^ciu variablo, kaj anstata^uigi per ^gi ^ciun apera^jon de tiu variablo.

Ekzemple, elkalkulante $3x^2 + (-2x) + 7$ je la elekto $x=4$, oni akiras \[
	3 \cdot \red{4}^2 + (-2 \cdot \red{4}) + 7 = 48 - 8 + 7 = 47.
\]
Ekde nun oni nur diskutas polinomojn en unu variablo, $x$.

\subsection{La kvadrata formulo}
\begin{mydef}
	\textbf{Radiko} de tia polinomo estas elekto por $x$, je kiu la rezulto de la elkalkulado estas $0$. (Ekzemple, 3 estas radiko de $x^2 - 9$, ^car $3^2 - 9 = 0$.)
\end{mydef}

\begin{mydef}
	\textbf Polinomo estas \textbf{kvadrata} se ^gi egalas $ax^2 + bx + c$ por iuj nombroj $a, b, c$ kun $a \neq 0$.
\end{mydef}

\begin{mythm}
	La radikoj $a x^2 + b x + c$ estas akirataj per la formulo: \[
		\boxed{x = \frac{-b \pm \sqrt{b^2-4ac}}{2a}.}
	\]
\end{mythm}

\begin{proof}
	En la ekvacio $a x^2 + b x + c = 0$, oni dividas amba^u flankon per $a$: \[
		x^2 + \frac{b}{a} x + \frac{c}{a} = 0.
	\]
	Nur oni adicias $\left( \frac{b}{2a} \right)^2 - \frac{c}{a}$ al amba^u flanko: \[
		x^2 + \frac{b}{a} x + \left( \frac{b}{2a} \right)^2 = \left( \frac{b}{2a} \right)^2 - \frac{c}{a}.
	\]
	Oni rekonas la livan flankon kiel $(x + \frac{b}{2a})^2$, kaj simpligas la dekstran. \[
		\left(x + \frac{b}{2a}\right)^2 = \frac{b^2 - 4ac}{4a^2}
	\]
	Oni scias ke $p^2 = q$ precize kiam $p = \pm \sqrt{q}$. Do konklude: \[
		x + \frac{b}{2a} = \sqrt{\frac{b^2 - 4ac}{4a^2}},
	\]
	kiun oni povas skribi kiel \[
		\boxed{x = \frac{-b \pm \sqrt{b^2 - 4ac}}{2a}.} \qedhere
	\]
\end{proof}
\end{document}
